\chapter{Are vulture restaurants needed to sustain the densest breeding population of the African White-backed Vulture? - Supplementary information}
\label{chap:pdp-supp}

%------------------------------------------	

\begin{table}[!htb]
\small %!htb keeps the table in this section before moving onto the next block of text
		\caption[Life history data] %This goes into  your list of tables
				{Life history parameters for all of the species in the community under study, values are in years \cite{brown1991declining,pennycuick1976breeding,jones2009pantheria,monadjem2012survival,piper1999modelling,de2009database}.} 
		\input{pdp_supp/tables/life_hist}
		\label{tab:life_hist}
	\end{table}

%------------------------------------------	

\begin{table}[H]
\small %!htb keeps the table in this section before moving onto the next block of text
		\caption[Avain scavenger reproductive parameters] %This goes into  your list of tables
				{Reproductive parameters for species of the avian scavenging guild (number of descendants corresponds to the number of fledglings for the following species) \cite{mundy1992vultures,monadjem2005nesting,mundy1982comparative,margalida2012modelling}.} 
		\input{pdp_supp/tables/repro_param}
		\label{tab:repro_param}
	\end{table}

%------------------------------------------	

\begin{table}[H]
\small %!htb keeps the table in this section before moving onto the next block of text
		\caption[Avain scavenger mortality parameters - adults] %This goes into  your list of tables
				{Mortality values for adults of the avian scavenging guild over a year \cite{brown1991declining,pennycuick1976breeding,monadjem2012survival,piper1999modelling,monadjem2013survival}.} 
		\input{pdp_supp/tables/adult_mort}
		\label{tab:adult_mort}
	\end{table}

%------------------------------------------	

\begin{table}[H]
\small %!htb keeps the table in this section before moving onto the next block of text
		\caption[Avain scavenger mortality parameters - immatures] %This goes into  your list of tables
				{Mortality values for immatures of the avian scavenging guild over a year \cite{brown1991declining,pennycuick1976breeding,monadjem2012survival,piper1999modelling,monadjem2013survival}.} 
		\input{pdp_supp/tables/imm_mort}
		\label{tab:imm_mort}
	\end{table}

%------------------------------------------



\begin{table}[H]
\small %!htb keeps the table in this section before moving onto the next block of text
		\caption[Carrion mass] %This goes into  your list of tables
				{Meat available for scavengers from carrion providing species \cite{jones2009pantheria,sinclair1995serengeti}.
} 
		\input{pdp_supp/tables/carrion_mass}
		\label{tab:carrion_mass}
	\end{table}

%------------------------------------------

\begin{table}
\small %!htb keeps the table in this section before moving onto the next block of text
		\caption[Population density of animals in ecosystem] %This goes into  your list of tables
				{Initial and maximum density of all the species considered in the ecosystem/ model. For the birds  we calculated the area potentially available to each species in Swaziland and then used recorded density estimates to determine the maximum number we would expect in such an area. Initial is equivalent to max density for the ungulates given our model assumptions mentioned above. The birds are in pairs, the ungulates are individuals \citep{monadjem2003threatened}.   
} 
		\input{pdp_supp/tables/pop_den}
		\label{tab:pop_den}
	\end{table}

%------------------------------------------

\begin{table}[H]
\small %!htb keeps the table in this section before moving onto the next block of text
		\caption[Avain scavenger food requirments] %This goes into  your list of tables
				{Daily Food Requirements for the avian scavenging guild. Marabou Stork food requirements are taken to be the same as the similarly sized Cape Griffon \cite{mundy1992vultures,mundy1982comparative,calder1996size}.  
} 
		\input{pdp_supp/tables/food_req}
		\label{tab:food_req}
	\end{table}
%------------------------------------------

\begin{table}[H]
\small %!htb keeps the table in this section before moving onto the next block of text
		\caption[Seasonal changes in the model] %This goes into  your list of tables
				{A summary of the changes in ungulate mortality, vulture food requirements and, foraging radius throughout the year and according to the periods considered.  
} 
		\input{pdp_supp/tables/season_change}
		\label{tab:season_change}
	\end{table}

%------------------------------------------

\begin{table}
\small %!htb keeps the table in this section before moving onto the next block of text
		\caption[Description of model parameters] %This goes into  your list of tables
				{Parameters and definitions used in the model.   
} 
		\input{pdp_supp/tables/param_desc}
		\label{tab:param_desc}
	\end{table}
%------------------------------------------


% PDP model
%\begin{figure}[h] %!htb keeps the figure in this section before moving onto the discussion
%	  \centering
%	  \includegraphics[keepaspectratio=true]{pdp_supp/figures/pdp_model.pdf}
%	    \caption[Full P Systems Model] %This is the label in table of contents
%	    {}%this is under the figure
%	  \label{fig:pdp_model}
%	\end{figure}
	
%------------------------------------------	
\newpage
\noindent {\textbf {PDP Model Components}} \\
Following \cite{colomer2013population}, there are four main components to the model:  1) A set of environments that are connected according to some prefixed relation, and which can be formally described by a network; 2) a membrane structure that provides the hierarchy among the different membranes that constitute the cell contained in each environment; 3) objects involved in the system under study (individuals, resources etc.) are represented by alphabetic characters; 4) A set of rules that specify the behaviour of the objects in the model and a set of rules for the model environments which define how individuals can move between the environments as well as generating values for the variables correlated between environments.

\newpage
\includepdf[pages={-}, ,scale=0.90]{pdp_supp/figures/pdp_model.pdf}
	\newpage

\newpage
\includepdf[pages={-}, ,scale=0.85]{pdp_supp/figures/paper.pdf}
	\newpage


