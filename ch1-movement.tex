

%\thesischapter{Movement ecology of the Cape Vulture}
%  {Adam Kane, Kerri Wolter, Walter Neser, Andrew L Jackson \& Ara Monadjem}

% \epigraph{Adam Kane, Kerri Wolter, Walter Neser, Ara Monadjem} can add an inspirational quote with this
\chapter{Movement ecology of the Cape Vulture}
\label{chap:introduction}

\textit{Authors:} Adam Kane, Kerri Wolter, Walter Neser, Andrew L Jackson \& Ara Monadjem
%Adam Kane$^1$, Kerri Wolter$^2$, Walter Neser$^2$, Andrew L Jackson$^1$ \& Ara Monadjem$^3$

\vspace{10 mm}
\noindent
\textit{\uppercase{A}uthor contributions}
I conceived the idea, analysed the data, interpreted the results and wrote the manuscript;
AM supplied data and advised on data analysis; 
AJ gave feedback on drafts of the manuscript;
KW and WN were responsible for the capture and tagging of the birds. 

\vspace{10 mm}

\noindent
\textit{Status:} This manuscript is being prepared for publication. Target journal is Bird Conservation International.

\newpage


\noindent

\section{\uppercase{A}bstract}

	Identifiying the areas animals use and how this varies across time is vital to conservation efforts esepcially in the case of wide-ranging vultures that traverse over international borders. Here we conducted a large scale analysis of the movement ecology of 29 Cape Vultures (\textit{Gyps coprotheres}) that were tracked for an average of 311 days over Southern Africa. We describe differences in home range size between adults and immatures as well as seasonal variation. We show that, in line with smaller scale studies, immature birds have larger home ranges than adults. This is likely due to strong competition at feeding sites near colonies and the constraint on adults of having dependent-young. There is also a significant effect of seasonality such that adult birds have smaller home ranges during the dry season. This is also likely a result of breeding birds being constrained by their nests as well as increased food availability provided by higher ungulate death rates. We discuss the conservation implications of these results with special mention given to vulture restaurants. We advocate the use of both frequently stocked restaurants close to colonies to benefit adults and infrequently stocked sites farther away from colonies for the benefit of immature birds. 
\newpage

\section{\uppercase{I}ntroduction}

	The Cape Vulture (\textit{Gyps coprotheres}) is a large (9 kg), obligate scavenging vulture known from southern Africa \citep{mundy1992vultures}. It is IUCN red listed as vulnerable with a declining population. Given their longevity and low reproductive output (clutch size is rarely more than one) Cape vultures are very sensitive to reductions in their survival rates \citep{phipps2013power,monadjem2014effect}. The bird is known to be a wide-ranging species \citep{bamford2007ranging}. Indeed \textit{Gyps} vultures \citep{monsarrat2013predictability} have a much larger home range than that of closely related members of the Accipitriformes \citep{peery2000factors} owing to their dependence on patchily distributed carrion \citep{ruxton2004obligate}. Such large home ranges make conserving these species a challenging task given that they may fly over many countries with different agendas pertaining to wildlife management \citep{lambertucci2014apex}. Seasonal and ontogenetic differences in home range have been illustrated before in \textit{Gyps} vultures \citep{monsarrat2013predictability, phipps2013power}. This is significant because specific life stages can have a disproportionate impact on the population growth rate, with the adult stage being most sensitive in the case of the Cape vulture \citep{monadjem2012survival,monadjem2014effect}.  \\
 \indent As such, the identification of home range is of great importance in conservation efforts, especially in recognising that home ranges are not static over either long or short periods \citep{burt1943territoriality}.  Here we define home range as "that area traversed by the individual in its normal activities of food gathering, mating, and caring for young" \citep{burt1943territoriality}. We used a large dataset of GPS tracked individuals, to test two intuitive predictions. First, we expected that the home range of immature birds would be larger than that of adults due to adults being constrained by their young \citep{mundy1992vultures}, the competition suffered by immature birds at feeding sites near colonies which forces them to forage farther afield and, the larger body mass of adults which can hamper ability to take off  \citep{robertson1986feeding,mundy1992vultures}. This is in line with previous studies based on fewer individuals \citep{mundy1992vultures, bamford2007ranging, phipps2013power}. Secondly, we hypothesised that adults would have a smaller home range during the breeding season compared to the non-breeding season, again because they have dependants as well as poor flying conditions during this period \citep{mundy1992vultures, monsarrat2013predictability}. The breeding season also coincides with the dry season for Cape vultures, this is a period during which the ungulates suffer an increased death rate and provide more carrion to the scavengers \citep{mundy1992vultures}. This is another factor that could reduce the home range of the birds given that they would encounter food more often during this period. 

Elucidating the movement ecology of these birds at different times, both seasonal and ontogenetic, is important because vulture conservation efforts frequently make use of long-term supplementary feeding sites \citep{gilbert2007vulture}. The benefits these sites may have could be maximised by knowing when the birds will most often avail of them and which life stages \citep{monsarrat2013predictability}. 



	%--------------------------------------
% Scatter-plot with Home Ranges

	
%------------------------------------------	
	

\section{\uppercase{M}ethods}
	In total, we analysed the movements of 29 Cape Vultures. These birds were captured by VulPro, a vulture conservation organisation, at trapping localities in South Africa using a walk-in trap \citep{diekmann2004capture,phipps2013power}. See \cite{phipps2013power} for details of their capture and release. Seven of these individuals have been used in a previous study investigating the effects of power lines on Cape Vulture movements \citep{phipps2013power}. The birds were separated into adult (> 5 years) and immature (< 5 years) categories \citep{piper1981estimates} (Table \ref{tab:homerange}). The sex of every bird was not known so we could not include it as an explanatory variable for home range variation; however it should be noted that adult \textit{Gyps} vultures share parental duties and would not be expected to vary much in this respect \citep{houston1976breeding}. 
 We calculated the home range for each individual using minimum convex polygon (MCP) \citep{mohr1947table} and kernel utilisation distribution methods (KUD) (both set to 95\%, such that 5\% of the most extreme points were removed)\citep{worton1989kernel}. We computed mean and standard deviation summary statistics for these home range areas for the individual birds. Then we used Wilcoxon-Mann-Whitney rank sum tests (the non-parametric alternative to the t-test) to determine if there was a difference in home range size between adult and immature Cape Vultures; and used the same method to test for a difference in home range with season (wet versus dry). The dry season was defined as May-August \citep{cooper1988foliage,mundy1992vultures}. Note that we did not know if a given adult was actually breeding. \\
 \indent For a subset of the birds, namely those that were tracked at a higher temporal resolution (once every 15 minutes cf. the adults who were tracked at 7.00, 11.00 and 15.00 each day), we estimated the distance they covered during a day (Table \ref{tab:dist_immature}). These were all immature individuals. We carried out our analysis using base R and the R package adehabitatHR \citep{RCran,calenge2013adehabitathr}. 

	%--------------------------------------
% Map with Home Ranges
\begin{figure}[H] %!htb keeps the figure in this section before moving onto the discussion
	  \centering
	  \includegraphics[width=0.7\textwidth]{chap1/figures/map.pdf}
	    \caption[Map of Cape Vulture distribution with home ranges] %This is the label in table of contents
	    {Distribution of the Cape Vulture in Southern Africa (data from Bird Life International in grey) and Home Range (KUD 95\%) of the 7 immature birds tracked at high resolution. }%this is under the figure
	  \label{fig:map}
	\end{figure}
	%,natwidth=610,natheight=642
%------------------------------------------	

\section{\uppercase{R}esults}
	Mean KUD home ranges were 282,255 km$^2$ (sd $\pm 288,460$ km$^2$) for immature birds and 110,181 km$^2$ (sd $\pm 130,464$ km$^2$) for adults (Table \ref{tab:homerange}). The Wilcoxon-Mann-Whitney rank sum tests showed this was a significant difference according to the KUD measure (p = 0.037). By comparison the MCP difference gave a p-value of 0.057 (figure \ref{fig:homerange}). The home ranges of the 7 high resolution birds are shown in figure \ref{fig:map}.

%-----------------------------------------------
%Home range figure	
%I'm not sure how to get the picture onto the same page as the table but there must be a way
	\begin{figure}[H] %!htb keeps the figure in this section before moving onto the discussion
	  \centering
	  \includegraphics[width=0.8\textwidth,natwidth=610,natheight=642]{chap1/figures/homerange.pdf}
	    \caption[Comparison of the home range of adult and immature birds. ] %This is the label in table of contents
	    {Comparison of the home range of adult and immature birds based on KUD. The median is shown by the horizontal bar within the box. The whole box captures the interquartile range of the home ranges i.e. the 25 to 75 
percentile. There is an outlier for both the adult and the immature birds.}%this is under the figure
	  \label{fig:homerange}
	\end{figure}

%--------------------------------------	

	Mean KUD for adults during wet and dry season were 80,719 km$^2$ (sd $\pm 137,799.1$ km$^2$) and 36,138 km$^2$ (sd $\pm 50,978$ km$^2$) respectively. Mean KUD for immatures during wet and dry season were 58,236 km$^2$ (sd $\pm 74,593$ km2) and 74621 km$^2$ (sd $\pm 99,304$ km$^2$) respectively. There was a significant difference between wet and dry season for adult birds (p = 0.03837) but not for immature birds (p = 0.5176) using KUD methods.

%------------------------------------------
	
	\begin{table}[H] %!htb keeps the table in this section before moving onto the next block of text
		\caption[Home range data] %This goes into  your list of tables
				{Home range (KUD and MCP) data on the Cape Vultures which were divided into immature and adult birds. Immature values are in the top half of the table, adults the bottom half.} %I used the HR abbreviation because otherwise the table was too wide for the page
		\input{chap1/tables/homerange}
		\label{tab:homerange}
	\end{table}
%-------------------------------------
	
	The average daily distance covered by the 7 high resolution birds was approximately 72 km if we used every relocation (Table \ref{tab:dist_immature}). However when we used only the start and end relocations for a given date, i.e. "as the crow flies", the average was reduced to just over 33 km as a result of the loss of smaller scale movements. 
	
%-------------------------------------	
	\begin{table}[H]
		\caption[Daily distance travelled by immature birds]
				{Average daily distance travelled by the immature birds that were tracked at a high resolution (once per 15 minutes) and the distance "as the crow flies" between the start and end points of the given date. }
		\input{chap1/tables/distance_immature}
		\label{tab:dist_immature}
	\end{table}

\vspace{10 mm}


\newpage
\section{\uppercase{D}iscussion}
The foraging day of a Cape vulture typically takes place between 10.00 and 18.00 (local time) \citep{mendelsohn2005observations}. They usually fly between an altitude of 250 - 350 m above ground \citep{mundy1992vultures}. The daily foraging distance reported here (an average of over 70 km) illustrates the impressive abilities of immature Cape Vultures in flight. In contrast, adults of the species have been calculated as covering a radius of 54 km in a day \citep{mundy1992vultures}. By way of comparison with immature vultures of other species, \cite{phipps2013foraging} showed immature African White-backed Vultures (\textit{Gyps africanus}) flew on average 33.39 km per day. Note that those estimates were based on locations recorded three times daily, a much coarser resolution, and thus nearly identical to our "as the crow flies" measure (mean = 33 km). This difference between adults and immatures accounts for the disparity seen in the home range of the Cape Vulture life stages. \\
 \indent
	Indeed, our large dataset on tracked Cape Vultures demonstrates a significant effect of both age and season on home range size. As mentioned above, these results are most probably the result of adults having to care for dependent young as well as the dominance hierarchy that exists across ontogeny \citep{duriez2012intra,mundy1992vultures}. We should also consider the effect of weather \citep{shepard2013daily}. The mass of an adult bird can be prohibitive when cold temperatures result in the formation of weak thermals. These cliff faces deflect wind upwards and are used by the soaring vultures to offset their sink rate \citep{shepard2013energy}. This can restrict adult flight time during winter months especially at colony cliffs in open savanna \citep{mundy1992vultures}. There is also the effect of differential seasonal mortality. The higher death rate of ungulates during the dry season means the vultures will encounter more carrion and won't have to travel as far. 

	\indent The difference in home range size between Cape Vulture life stages has implications for its conservation. We know from population growth models showing that adults have a disproportionate effect on the growth rate of Cape vulture populations \citep{monadjem2014effect}. Habitat fragmentation is known to increase the size of an animal's home range because it causes a reduction in resource density \citep{haskell2002fractal}. To counter this, conservationists can supply supplementary food \citep{piper2005supplementary}. However, these sites impact adult and immature birds in different ways depending on how they are run.  \cite{duriez2012intra} showed that, at artificial feeding sites, close to colonies and regularly stocked with food, adult vultures (\textit{Gyps fulvus}) dominated young individuals who were left to fight over scraps. But "light" feeding sites, located farther from colonies and supplied less regularly, were preferred by immature birds. Conservationists should therefore use a mix of "light" and "heavy" feeding sites if they are to effectively manage both ontogenetic stages. \cite{komen1991energy} calculated the energy requirements for this species across its ontogeny, a piece of research that could be put to good use when supplying supplementary feeding sites at different times. We also have excellent data on the population sizes of Cape Vulture colonies in South Africa \citep{whittington2011monitoring}. By knowing where they forage, how much food they require and variability in the species home range we are well placed to sustain this species. However, conservation actions must be at an international level \citep{lambertucci2014apex} given that Cape Vultures are not impeded by national boundaries. 


%\end{document}
